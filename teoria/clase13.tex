\documentclass[handout]{beamer}
%%%%%%%%%%%%%%%%%%%%%%%%%% DO NOT MODIFY %%%%%%%%%%%%%%%%%%%%%%%%%
%% preamble
\usetheme{Madrid}
\mode<presentation>
\usepackage[spanish]{babel}
\usepackage[utf8]{inputenc}
\usepackage{lmodern}
\usepackage{hyperref}
\usepackage{listings}
\usepackage[T1]{fontenc}

\defbeamertemplate*{footline}{shadow theme}{%
  \leavevmode%
  \hbox{\begin{beamercolorbox}[wd=.5\paperwidth,ht=2ex,dp=1ex,leftskip=.3cm plus1fil,rightskip=.3cm]{author in head/foot}%
      \usebeamerfont{author in head/foot}\hfill\insertshortauthor
  \end{beamercolorbox}%

  \begin{beamercolorbox}[wd=.5\paperwidth,ht=2ex,dp=1ex,leftskip=.3cm,rightskip=.3cm plus1fil]{title in head/foot}%
      \usebeamerfont{title in head/foot}\insertshorttitle\hfill%
  \insertframenumber\,/\,\inserttotalframenumber
  \end{beamercolorbox}}%
  \vskip0pt%
}

\AtBeginSection[]{
  \begin{frame}[allowframebreaks]{Contenido}
  \small{\tableofcontents[currentsection]}
  \end{frame}
}

\AtBeginSubsection[]{
  \begin{frame}[allowframebreaks]{Contenido}
    \small{\tableofcontents[currentsubsection]}
  \end{frame}
}
\beamertemplatenavigationsymbolsempty 
\usefonttheme[onlymath]{serif}

\lstdefinestyle{customHTML}{
  belowcaptionskip=1\baselineskip,
  breaklines=true,
  frame=L,
  xleftmargin=\parindent,
  language=HTML,
  showstringspaces=false,
  basicstyle=\footnotesize\ttfamily,
  keywordstyle=\bfseries\color{blue},
  commentstyle=\itshape\color{gray},
  identifierstyle=\color{black},
  stringstyle=\color{orange},
}
\lstset{escapechar=@,style=customHTML}

%%%%%%%%%%%%%%%%%%%%%%%%% END DO NOT MODIFY %%%%%%%%%%%%%%%%%%%%

\title{Introducción a la Programación}
\subtitle{Ejercicios Estructuras de repetición}
\author{Edwin Salvador}
\date{12 de julio de 2016}
\begin{document}
  \begin{frame}
    \titlepage
    \centerline{Clase 13}
  \end{frame}
  
  \AtBeginSection[]{
    \begin{frame}[allowframebreaks]{Contenido}
      \tableofcontents[currentsection]
    \end{frame}
  }

\section{Contador} % (fold)
\label{sec:contador}
\begin{frame}[fragile]\frametitle{Contador}
    
Una variable cuyo valor se irá incrementando manualmente.

\begin{lstlisting}
x = x + 1;    
\end{lstlisting}
Ejemplo: Si \texttt{x} es 2, luego de hacer \texttt{x = x + 1;} el valor de \texttt{x} será 3.

\end{frame}

\begin{frame}[t]\frametitle{Ejercicio}
    
Resolver el problema utilizando un \texttt{while} con un contador.

Mostrar por pantalla los primeros n números naturales considerando al 0 (cero) como primer número natural.

\end{frame}
% section contador (end)

\section{Acumulador} % (fold)
\label{sec:acumulador}
\begin{frame}[fragile]\frametitle{Acumulador}
    
Variable cuyo valor iremos incrementando en cantidades variables dentro de un ciclo de repetición. 
\begin{lstlisting}
    x = x + n;
\end{lstlisting}

\end{frame}

\begin{frame}[t]\frametitle{Ejercicio}
    
Determinar la sumatoria de los elementos de un conjunto de valores numéricos. Los números se ingresarán por teclado. Se ingresará un cero para  finalizar.


\end{frame}
% section acumulador (end)

\section{Ejercicios} % (fold)
\label{sec:ejercicios}

\begin{frame}[t]\frametitle{Ejercicio}
    
Se ingresa un valor numérico por consola, determinar e informar si se trata de un número primo o no.

\end{frame}

\begin{frame}[t]\frametitle{Ejercicio}
    
Dado un conjunto de valores numéricos indicar cuál es el mayor. El ingreso de datos  finaliza con la llegada de un cero.

\end{frame}
% section ejercicios (end)

\section{Deber} % (fold)
\label{sec:deber}
\begin{frame}[t]\frametitle{Deber}
    
\begin{itemize}
    \item Determinar el menor valor de un conjunto de números e indicar también su posición relativa dentro del mismo. El ingreso de datos finaliza con la llegada de un cero. El programa debe presentar el número menor y la posición cuando se termina de ingresar los números.
    \item Desarrollar un algoritmo que muestre los primeros n números primos siendo n un valor que debe ingresar el usuario.
    \item Dado un conjunto de valores numéricos que se ingresan por teclado determinar el valor promedio. El  n de datos se indicará ingresando un valor igual a cero.
\end{itemize}


\end{frame}
% section deber (end)

\end{document}
